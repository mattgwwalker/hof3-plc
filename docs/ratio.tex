\documentclass{article}

\begin{document}

\section*{Calculating the Concentration Ratio}

To calculate the concentration ratio, $r_t$, of the plant's current contents, we measure (for any time $t$):
\begin{itemize}
\item The volume of product in the system, $v_t$
\item The flow of permeate, $f_2,t$
\item The flow of retentate, $f_3,t$
\end{itemize}

The volume of permeate, $v_{2,t}$ can be calculated from the flow rate and the duration of time, $\Delta t$:
\begin{equation}
  v_{2,t} = f_{2,t} \cdot \Delta t
\end{equation}

In a similar way, we can calculate the volume of retentate:
\begin{equation}
  v_{3,t} = f_{3,t} \cdot \Delta t
\end{equation}


From those measurements we can calculate the volume going into the system
\begin{equation}
  v_{1,t} = v_{t} - v_{t-1} + v_{2,t} + v_{3,t}
  \label{eq:v1}
\end{equation}

To make the calculation for the concentration ratio, we must consider the mass of solids.

We begin by saying that product going into the tank has a concentration ratio of one (i.e.~it is full-strength product).  This product has a solids content of $C$.  It is unimportant what units are used for $C$, but for instance it could be measured as grams per litre.

We wish to know the mass of solids in the system, $m_t$, at time $t$.  This can be calculated from the volume of liquid in the system, its current concentration ratio and the solids content of full-strength product:
\begin{equation}
  m_t = v_t \cdot r_t \cdot C
\end{equation}

Equally, the mass of the system at time $t-1$ can be calculated:
\begin{equation}
  m_{t-1} = v_{t-1} \cdot r_{t-1} \cdot C
\end{equation}

The mass of solids going into the system ($m_1$) is the volume going in ($v_{1,t}$), multiplied by the ratio of full-strength product (one), multiplied by the solids content of full-strength product ($C$):
\begin{equation}
  m_{1,t} = v_{1,t}  \cdot C
\end{equation}

The mass of solids going out via the retentate line, $m_3$ is the product of the volume out of the retentate line, the concentration ratio of the system and the solids content of full-strength product.  We use $r_{t-1}$ as a convenient   approximation for the concentration ratio.
\begin{equation}
  m_{3,t} = v_{3,t}  \cdot r_{t-1} \cdot C
\end{equation}

The mass of solids going out the permeate line, $m_2$, is the product of the volume out of the permeate line, the concentration ratio of the system, the solids content of full-strength product and a $k$ factor which designates the proportion of solids that pass via the permeate line.  A $k$ factor of 0 would mean that no solids permeate the membrane, while a $k$ factor of one would indicate the membrane was not filtering the product.  It is possible the $k$-factor is dependent on the current concentration ratio, but this effect will not be considered.
\begin{equation}
  m_{2,t} = v_{2,t}  \cdot r_{t-1} \cdot C \cdot k
\end{equation}

The solids content of the system at time $t$ is the solids content of the system at time $t-1$ plus the solids added into the system, $m_{1,t}$, less the solids removed via the permeate line $m_{2,t}$, less the solids removed via the retentate line $m_{3,t}$:
\begin{equation}
  m_t = m_{t-1} + m_{1,t} - m_{2,t} - m_{3,t}
\end{equation}

Replacing the masses with their respective volume-based equivalents gives:
\begin{equation}
  m_t = v_t \cdot r_t \cdot C
  	= v_{t-1} \cdot r_{t-1} \cdot C 
   	+ v_{1,t}  \cdot C
	- v_{2,t}  \cdot r_{t-1} \cdot C \cdot k
	- v_{3,t}  \cdot r_{t-1} \cdot C
\end{equation}

Eliminating $C$ gives:
\begin{equation}
  v_t \cdot r_t
  	= v_{t-1} \cdot r_{t-1}
   	+ v_{1,t}
	- v_{2,t}  \cdot r_{t-1} \cdot k
	- v_{3,t}  \cdot r_{t-1}
\end{equation}

We then replace $v_{1,t}$ with what we are measuring (from equation~\ref{eq:v1}):
\begin{equation}
  v_t \cdot r_t
  	= v_{t-1} \cdot r_{t-1}
	+ v_{t} - v_{t-1} + v_{2,t} + v_{3,t}
	- v_{2,t}  \cdot r_{t-1} \cdot k
	- v_{3,t}  \cdot r_{t-1}
\end{equation}

And then factor out the repeating expressions to get:
\begin{equation}
  v_t \cdot r_t
  	= v_{t-1} \cdot (r_{t-1} -1)
	+ v_{t} + v_{2,t}(1 - r_{t-1} \cdot k)  
	+ v_{3,t}(1 - r_{t-1})
\end{equation}

Divide both sides by $v_t$ gives:
\begin{equation}
  r_t
  	= 1+\frac{v_{t-1} \cdot (r_{t-1} -1)
	+ v_{2,t}(1 - r_{t-1} \cdot k)  
	+ v_{3,t}(1 - r_{t-1})}{v_t}
\end{equation}


\end{document}